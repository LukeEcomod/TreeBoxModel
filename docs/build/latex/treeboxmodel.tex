%% Generated by Sphinx.
\def\sphinxdocclass{report}
\documentclass[letterpaper,10pt,english]{sphinxmanual}
\ifdefined\pdfpxdimen
   \let\sphinxpxdimen\pdfpxdimen\else\newdimen\sphinxpxdimen
\fi \sphinxpxdimen=.75bp\relax

\PassOptionsToPackage{warn}{textcomp}
\usepackage[utf8]{inputenc}
\ifdefined\DeclareUnicodeCharacter
% support both utf8 and utf8x syntaxes
  \ifdefined\DeclareUnicodeCharacterAsOptional
    \def\sphinxDUC#1{\DeclareUnicodeCharacter{"#1}}
  \else
    \let\sphinxDUC\DeclareUnicodeCharacter
  \fi
  \sphinxDUC{00A0}{\nobreakspace}
  \sphinxDUC{2500}{\sphinxunichar{2500}}
  \sphinxDUC{2502}{\sphinxunichar{2502}}
  \sphinxDUC{2514}{\sphinxunichar{2514}}
  \sphinxDUC{251C}{\sphinxunichar{251C}}
  \sphinxDUC{2572}{\textbackslash}
\fi
\usepackage{cmap}
\usepackage[T1]{fontenc}
\usepackage{amsmath,amssymb,amstext}
\usepackage{babel}



\usepackage{times}
\expandafter\ifx\csname T@LGR\endcsname\relax
\else
% LGR was declared as font encoding
  \substitutefont{LGR}{\rmdefault}{cmr}
  \substitutefont{LGR}{\sfdefault}{cmss}
  \substitutefont{LGR}{\ttdefault}{cmtt}
\fi
\expandafter\ifx\csname T@X2\endcsname\relax
  \expandafter\ifx\csname T@T2A\endcsname\relax
  \else
  % T2A was declared as font encoding
    \substitutefont{T2A}{\rmdefault}{cmr}
    \substitutefont{T2A}{\sfdefault}{cmss}
    \substitutefont{T2A}{\ttdefault}{cmtt}
  \fi
\else
% X2 was declared as font encoding
  \substitutefont{X2}{\rmdefault}{cmr}
  \substitutefont{X2}{\sfdefault}{cmss}
  \substitutefont{X2}{\ttdefault}{cmtt}
\fi


\usepackage[Bjarne]{fncychap}
\usepackage{sphinx}

\fvset{fontsize=\small}
\usepackage{geometry}


% Include hyperref last.
\usepackage{hyperref}
% Fix anchor placement for figures with captions.
\usepackage{hypcap}% it must be loaded after hyperref.
% Set up styles of URL: it should be placed after hyperref.
\urlstyle{same}
\addto\captionsenglish{\renewcommand{\contentsname}{Details of the model}}

\usepackage{sphinxmessages}
\setcounter{tocdepth}{1}



\title{Tree Box Model}
\date{May 26, 2020}
\release{0.3}
\author{Olli\sphinxhyphen{}Pekka Tikkanen}
\newcommand{\sphinxlogo}{\vbox{}}
\renewcommand{\releasename}{Release}
\makeindex
\begin{document}

\pagestyle{empty}
\sphinxmaketitle
\pagestyle{plain}
\sphinxtableofcontents
\pagestyle{normal}
\phantomsection\label{\detokenize{index::doc}}



\chapter{Description of the modelled system}
\label{\detokenize{modelled_system:description-of-the-modelled-system}}\label{\detokenize{modelled_system::doc}}
Modelled system is a tree


\chapter{Instructions to run the model}
\label{\detokenize{instructions_to_run:instructions-to-run-the-model}}\label{\detokenize{instructions_to_run::doc}}
There are two options to run the model


\section{Running from main.py}
\label{\detokenize{instructions_to_run:running-from-main-py}}

\section{Importing src.model}
\label{\detokenize{instructions_to_run:importing-src-model}}

\chapter{Installation}
\label{\detokenize{index:installation}}
\begin{sphinxVerbatim}[commandchars=\\\{\}]
\PYG{g+gp}{\PYGZgt{}\PYGZgt{}\PYGZgt{} }\PYG{n}{git} \PYG{n}{clone} \PYG{n}{git}\PYG{n+nd}{@github}\PYG{o}{.}\PYG{n}{com}\PYG{p}{:}\PYG{n}{LukeEcomod}\PYG{o}{/}\PYG{n}{TreeBoxModel}\PYG{o}{.}\PYG{n}{git}
\end{sphinxVerbatim}

or

download the source \sphinxurl{https://github.com/LukeEcomod/TreeBoxModel}


\chapter{Quick start}
\label{\detokenize{index:quick-start}}
run main.py

\begin{sphinxVerbatim}[commandchars=\\\{\}]
\PYG{g+gp}{\PYGZgt{}\PYGZgt{}\PYGZgt{} }\PYG{n}{python} \PYG{n}{main}\PYG{o}{.}\PYG{n}{py}
\end{sphinxVerbatim}


\chapter{main.py}
\label{\detokenize{index:module-main}}\label{\detokenize{index:main-py}}\index{module@\spxentry{module}!main@\spxentry{main}}\index{main@\spxentry{main}!module@\spxentry{module}}
The purpose of this main file is to provide an easy way to run the model.

All the model parameters are set in the file and are taken from \sphinxhref{https://link.springer.com/article/10.1007/s00468-005-0014-6}{Hölttä et. al. 2006}
or \sphinxhref{https://academic.oup.com/aob/article/114/4/653/2769025}{Nikinmaa et. al., (2014)}.

A sine\sphinxhyphen{}like behaviour is assumed for the transpiration and photosynthesis

\sphinxhref{../../source/\_static/transpiration\_rate.png}{\sphinxincludegraphics{{../../source/_static/transpiration_rate}.png}}


\chapter{Modules, Classes \& functions}
\label{\detokenize{index:module-src.model}}\label{\detokenize{index:modules-classes-functions}}\index{module@\spxentry{module}!src.model@\spxentry{src.model}}\index{src.model@\spxentry{src.model}!module@\spxentry{module}}\index{Model (class in src.model)@\spxentry{Model}\spxextra{class in src.model}}

\begin{fulllineitems}
\phantomsection\label{\detokenize{index:src.model.Model}}\pysiglinewithargsret{\sphinxbfcode{\sphinxupquote{class }}\sphinxcode{\sphinxupquote{src.model.}}\sphinxbfcode{\sphinxupquote{Model}}}{\emph{\DUrole{n}{tree}\DUrole{p}{:} \DUrole{n}{{\hyperref[\detokenize{index:src.tree.Tree}]{\sphinxcrossref{src.tree.Tree}}}}}, \emph{\DUrole{n}{outputfile}\DUrole{p}{:} \DUrole{n}{str} \DUrole{o}{=} \DUrole{default_value}{\textquotesingle{}a.nc\textquotesingle{}}}}{}
Calculates the next time step for given tree and saves the tree stage.

Provides functionality for solving the ordinary differential equations (ODE) describing
the behaviour of the \sphinxhref{modelled\_system.html}{modelled system}.
\begin{quote}\begin{description}
\item[{Parameters}] \leavevmode\begin{itemize}
\item {} 
\sphinxstyleliteralstrong{\sphinxupquote{tree}} ({\hyperref[\detokenize{index:src.tree.Tree}]{\sphinxcrossref{\sphinxstyleliteralemphasis{\sphinxupquote{Tree}}}}}) \textendash{} instance of the tree class for which the ODEs are solved

\item {} 
\sphinxstyleliteralstrong{\sphinxupquote{outputfile}} (\sphinxstyleliteralemphasis{\sphinxupquote{str}}) \textendash{} name of the file where the NETCDF4 output is written

\end{itemize}

\end{description}\end{quote}
\index{tree (src.model.Model attribute)@\spxentry{tree}\spxextra{src.model.Model attribute}}

\begin{fulllineitems}
\phantomsection\label{\detokenize{index:src.model.Model.tree}}\pysigline{\sphinxbfcode{\sphinxupquote{tree}}}
instance of the tree class for which the ODEs are solved
\begin{quote}\begin{description}
\item[{Type}] \leavevmode
{\hyperref[\detokenize{index:src.tree.Tree}]{\sphinxcrossref{Tree}}}

\end{description}\end{quote}

\end{fulllineitems}

\index{outputfile (src.model.Model attribute)@\spxentry{outputfile}\spxextra{src.model.Model attribute}}

\begin{fulllineitems}
\phantomsection\label{\detokenize{index:src.model.Model.outputfile}}\pysigline{\sphinxbfcode{\sphinxupquote{outputfile}}}
name of the file where the output is written
\begin{quote}\begin{description}
\item[{Type}] \leavevmode
str

\end{description}\end{quote}

\end{fulllineitems}

\index{ncf (src.model.Model attribute)@\spxentry{ncf}\spxextra{src.model.Model attribute}}

\begin{fulllineitems}
\phantomsection\label{\detokenize{index:src.model.Model.ncf}}\pysigline{\sphinxbfcode{\sphinxupquote{ncf}}}
the output file
\begin{quote}\begin{description}
\item[{Type}] \leavevmode
netCDF4.Dataset

\end{description}\end{quote}

\end{fulllineitems}

\index{axial\_fluxes() (src.model.Model method)@\spxentry{axial\_fluxes()}\spxextra{src.model.Model method}}

\begin{fulllineitems}
\phantomsection\label{\detokenize{index:src.model.Model.axial_fluxes}}\pysiglinewithargsret{\sphinxbfcode{\sphinxupquote{axial\_fluxes}}}{}{{ $\rightarrow$ numpy.ndarray}}
Calculates axial sap mass flux for every element.

The axial flux in the xylem and phloem are calculated independently from the sum of bottom and top fluxes
\begin{equation*}
\begin{split}Q_{ax,i} = Q_{ax,bottom,i} + Q_{ax,top,i}\end{split}
\end{equation*}\begin{align*}\!\begin{aligned}
Q_{ax,bottom,i} = \frac{k_i \: A_{ax,i} \: \rho_w}{\eta_i \: l_i}(P_{i+1} - P{i} - P_h)\\
Q_{ax,top,i} = \frac{k_i \: A_{ax,i+1} \: \rho_w}{\eta_i \: l_i}(P_{i-1} - P{i} + P_h)\\
\end{aligned}\end{align*}
where

\(k_i\): axial permeability of the ith element (\(m^2\))
\(A_{ax,i}\): base surface area of xylem or phloem (\(m^2\))
\(\rho_w\): liquid phase density of water (\(\frac{kg}{m^3}\))
\(\eta\): viscosity of the sap in the ith element (\(Pa \: s\))
\(l_i\): length (height) of the ith element (\(m\))
\(P_{i}\): Pressure in the ith element (\(Pa\))
\(P_h\): Hydrostatic pressure (\(Pa\)) \(P_h = \rho_w a_{gravitation} l_i\)
\begin{quote}\begin{description}
\item[{Returns}] \leavevmode
The axial fluxes in units kg/s

\item[{Return type}] \leavevmode
numpy.ndarray (dtype=float, ndim=2){[}self.tree.num\_elements, 2{]}

\end{description}\end{quote}

\end{fulllineitems}

\index{radial\_fluxes() (src.model.Model method)@\spxentry{radial\_fluxes()}\spxextra{src.model.Model method}}

\begin{fulllineitems}
\phantomsection\label{\detokenize{index:src.model.Model.radial_fluxes}}\pysiglinewithargsret{\sphinxbfcode{\sphinxupquote{radial\_fluxes}}}{}{{ $\rightarrow$ numpy.ndarray}}
Calculates radial sap mass flux for every element.

The radial flux for the phloem of the ith axial is calculated similar to
\sphinxhref{https://link.springer.com/article/10.1007/s00468-005-0014-6}{Hölttä et. al. 2006}
\begin{equation*}
\begin{split}Q_{radial,phloem} = L_r A_{rad,i}\rho_{w}
[P_{i,xylem} - P_{i,phloem} - \sigma(C_{i,xylem} - C_{i,phloem})RT)]\end{split}
\end{equation*}
where

\(L_r\): radial hydraulic conductivity (\(\frac{m}{Pa \: s}\))
\(A_{rad,i}\): lateral surface area of the xylem (\(m^2\))
\(\rho_w\): liquid phase density of water (\(\frac{kg}{m^3}\))
\(P_{i}\): Pressure in the ith element (\(Pa\))
\(\sigma\): Reflection coefficient (Van’t hoff factor) (unitless)
\(C_{i}\): Sucrose concentration in the ith element (\(\frac{mol}{m^3}\))
\(R\): Universal gas constant (\(\frac{J}{K \: mol}\))
\(T\): Ambient temperature (\(K\))

The radial flux for the xylem is equal to the additive inverse of the phloem flux
\begin{equation*}
\begin{split}Q_{radial,xylem} = -Q_{radial,phloem}\end{split}
\end{equation*}\begin{quote}\begin{description}
\item[{Returns}] \leavevmode
The radial fluxes in units kg/s

\item[{Return type}] \leavevmode
numpy.ndarray (dtype=float, ndim=2){[}self.tree.num\_elements, 2{]}

\end{description}\end{quote}

\end{fulllineitems}

\index{run() (src.model.Model method)@\spxentry{run()}\spxextra{src.model.Model method}}

\begin{fulllineitems}
\phantomsection\label{\detokenize{index:src.model.Model.run}}\pysiglinewithargsret{\sphinxbfcode{\sphinxupquote{run}}}{\emph{\DUrole{n}{time\_start}\DUrole{p}{:} \DUrole{n}{float} \DUrole{o}{=} \DUrole{default_value}{0.001}}, \emph{\DUrole{n}{time\_end}\DUrole{p}{:} \DUrole{n}{float} \DUrole{o}{=} \DUrole{default_value}{120.0}}, \emph{\DUrole{n}{dt}\DUrole{p}{:} \DUrole{n}{float} \DUrole{o}{=} \DUrole{default_value}{0.01}}, \emph{\DUrole{n}{output\_interval}\DUrole{p}{:} \DUrole{n}{float} \DUrole{o}{=} \DUrole{default_value}{60}}}{{ $\rightarrow$ None}}
Propagates the tree in time using explicit Euler method (very slow).

NB! This function needs to be updated. Use run\_scipy instead!
\begin{quote}\begin{description}
\item[{Parameters}] \leavevmode\begin{itemize}
\item {} 
\sphinxstyleliteralstrong{\sphinxupquote{time\_start}} (\sphinxstyleliteralemphasis{\sphinxupquote{float}}) \textendash{} Time in seconds where to start the simulation.

\item {} 
\sphinxstyleliteralstrong{\sphinxupquote{time\_ned}} (\sphinxstyleliteralemphasis{\sphinxupquote{float}}) \textendash{} Time in seconds where to end the simulation.

\item {} 
\sphinxstyleliteralstrong{\sphinxupquote{dt}} (\sphinxstyleliteralemphasis{\sphinxupquote{float}}) \textendash{} time step in seconds

\item {} 
\sphinxstyleliteralstrong{\sphinxupquote{output\_interval}} \textendash{} Time interval in seconds when to save the tree stage

\end{itemize}

\end{description}\end{quote}

\end{fulllineitems}

\index{run\_scipy() (src.model.Model method)@\spxentry{run\_scipy()}\spxextra{src.model.Model method}}

\begin{fulllineitems}
\phantomsection\label{\detokenize{index:src.model.Model.run_scipy}}\pysiglinewithargsret{\sphinxbfcode{\sphinxupquote{run\_scipy}}}{\emph{\DUrole{n}{time\_start}\DUrole{p}{:} \DUrole{n}{float} \DUrole{o}{=} \DUrole{default_value}{0.001}}, \emph{\DUrole{n}{time\_end}\DUrole{p}{:} \DUrole{n}{float} \DUrole{o}{=} \DUrole{default_value}{120.0}}, \emph{\DUrole{n}{ind}\DUrole{p}{:} \DUrole{n}{int} \DUrole{o}{=} \DUrole{default_value}{0}}}{{ $\rightarrow$ None}}
Propagates the tree in time using the solve\_ivp function in the SciPy package.

The stage of the tree is saved only at the start of the simulation if time\_start \textless{} 1e\sphinxhyphen{}3 and at time\_end.
If the tree stage is desired on multiple time points the function needs to be called recurrently by splitting
the time interval into multiple sub intervals.
\begin{quote}\begin{description}
\item[{Parameters}] \leavevmode\begin{itemize}
\item {} 
\sphinxstyleliteralstrong{\sphinxupquote{time\_start}} (\sphinxstyleliteralemphasis{\sphinxupquote{float}}) \textendash{} Time in seconds where to start the simulation.

\item {} 
\sphinxstyleliteralstrong{\sphinxupquote{time\_ned}} (\sphinxstyleliteralemphasis{\sphinxupquote{float}}) \textendash{} Time in seconds where to end the simulation.

\item {} 
\sphinxstyleliteralstrong{\sphinxupquote{ind}} (\sphinxstyleliteralemphasis{\sphinxupquote{int}}) \textendash{} index which refers to the index in model.outputfile.
The last stage of the tree is saved to model.outputfile{[}ind{]}.

\end{itemize}

\end{description}\end{quote}

\end{fulllineitems}


\end{fulllineitems}

\phantomsection\label{\detokenize{index:module-src.tree}}\index{module@\spxentry{module}!src.tree@\spxentry{src.tree}}\index{src.tree@\spxentry{src.tree}!module@\spxentry{module}}\index{Tree (class in src.tree)@\spxentry{Tree}\spxextra{class in src.tree}}

\begin{fulllineitems}
\phantomsection\label{\detokenize{index:src.tree.Tree}}\pysiglinewithargsret{\sphinxbfcode{\sphinxupquote{class }}\sphinxcode{\sphinxupquote{src.tree.}}\sphinxbfcode{\sphinxupquote{Tree}}}{\emph{\DUrole{n}{height}\DUrole{p}{:} \DUrole{n}{float}}, \emph{\DUrole{n}{initial\_radius}\DUrole{p}{:} \DUrole{n}{List\DUrole{p}{{[}}float\DUrole{p}{{]}}}}, \emph{\DUrole{n}{num\_elements}\DUrole{p}{:} \DUrole{n}{int}}, \emph{\DUrole{n}{transpiration\_profile}\DUrole{p}{:} \DUrole{n}{List\DUrole{p}{{[}}float\DUrole{p}{{]}}}}, \emph{\DUrole{n}{photosynthesis\_profile}\DUrole{p}{:} \DUrole{n}{List\DUrole{p}{{[}}float\DUrole{p}{{]}}}}, \emph{\DUrole{n}{sugar\_profile}\DUrole{p}{:} \DUrole{n}{List\DUrole{p}{{[}}float\DUrole{p}{{]}}}}, \emph{\DUrole{n}{sugar\_loading\_profile}\DUrole{p}{:} \DUrole{n}{List\DUrole{p}{{[}}float\DUrole{p}{{]}}}}, \emph{\DUrole{n}{sugar\_unloading\_profile}\DUrole{p}{:} \DUrole{n}{List\DUrole{p}{{[}}float\DUrole{p}{{]}}}}, \emph{\DUrole{n}{sugar\_target\_concentration}\DUrole{p}{:} \DUrole{n}{float}}, \emph{\DUrole{n}{sugar\_unloading\_slope}\DUrole{p}{:} \DUrole{n}{float}}, \emph{\DUrole{n}{axial\_permeability\_profile}\DUrole{p}{:} \DUrole{n}{List\DUrole{p}{{[}}List\DUrole{p}{{[}}float\DUrole{p}{{]}}\DUrole{p}{{]}}}}, \emph{\DUrole{n}{radial\_hydraulic\_conductivity\_profile}\DUrole{p}{:} \DUrole{n}{List\DUrole{p}{{[}}float\DUrole{p}{{]}}}}, \emph{\DUrole{n}{elastic\_modulus\_profile}\DUrole{p}{:} \DUrole{n}{List\DUrole{p}{{[}}List\DUrole{p}{{[}}float\DUrole{p}{{]}}\DUrole{p}{{]}}}}, \emph{\DUrole{n}{ground\_water\_potential}\DUrole{p}{:} \DUrole{n}{float}}}{}
Model of a tree.

Provides properties and functionality for saving and editing the modelled tree. Arguments whose
type is List{[}float{]} or List{[}List{[}float{]}{]} are converted to numpy.ndarray with numpy.asarray method.
Thus, also numpy.ndarray is a valid type for these arguments.

For arguemnts whose type is List{[}float{]} (except for initial\_radius) the length of the arguments must
be equal to num\_elements. The order of the list should be from the top of the tree (the first item) to the
bottom of the tree (the last item)

For arguments whose type is List{[}List{[}float{]}{]} the length of the arguemnts must be equal to num\_elements
and each sub list must contain two elements, one for the xylem and one for the phloem in this order. The
order of the sub lists should be from the top of the tree (the first sub list) to the bottom of the tree
(the last sub list).
\begin{quote}\begin{description}
\item[{Parameters}] \leavevmode\begin{itemize}
\item {} 
\sphinxstyleliteralstrong{\sphinxupquote{height}} (\sphinxstyleliteralemphasis{\sphinxupquote{float}}) \textendash{} total tree height (\(m\))

\item {} 
\sphinxstyleliteralstrong{\sphinxupquote{initial\_radius}} (\sphinxstyleliteralemphasis{\sphinxupquote{List}}\sphinxstyleliteralemphasis{\sphinxupquote{{[}}}\sphinxstyleliteralemphasis{\sphinxupquote{float}}\sphinxstyleliteralemphasis{\sphinxupquote{{]} or }}\sphinxstyleliteralemphasis{\sphinxupquote{numpy.ndarray}}) \textendash{} 
the radius of the xylem and the phloem (\(m\)) in this order.
See from the \sphinxhref{modelled\_system.html}{modelled system}, how the radii should be given. Only two values can
be given and the radius of each element is set to be the same in the tree initialization.


\item {} 
\sphinxstyleliteralstrong{\sphinxupquote{num\_elements}} (\sphinxstyleliteralemphasis{\sphinxupquote{int}}) \textendash{} number of vertical elemenets in the tree.
The height of an element is determined by
\(\text{element height} = \frac{\text{tree height}}{\text{number of elements}}\)

\item {} 
\sphinxstyleliteralstrong{\sphinxupquote{transpiration\_profile}} (\sphinxstyleliteralemphasis{\sphinxupquote{List}}\sphinxstyleliteralemphasis{\sphinxupquote{{[}}}\sphinxstyleliteralemphasis{\sphinxupquote{float}}\sphinxstyleliteralemphasis{\sphinxupquote{{]} or }}\sphinxstyleliteralemphasis{\sphinxupquote{numpy.ndarray}}) \textendash{} The rate of transpiration (\(\frac{kg}{s}\)) in the
xylem. The length of the list must be equal to num\_elements and the order is from the top of the tree
(first value) in the list to the bottom of the tree (last value in the list).

\item {} 
\sphinxstyleliteralstrong{\sphinxupquote{photosynthesis\_profile}} (\sphinxstyleliteralemphasis{\sphinxupquote{List}}\sphinxstyleliteralemphasis{\sphinxupquote{{[}}}\sphinxstyleliteralemphasis{\sphinxupquote{float}}\sphinxstyleliteralemphasis{\sphinxupquote{{]}}}) \textendash{} The rate of photosynthesis (\(\frac{mol}{s}\)). Currently this
variable is not used and the rate of photosynthesis should be equal to the sugar\_loading\_profile.

\item {} 
\sphinxstyleliteralstrong{\sphinxupquote{sugar\_profile}} (\sphinxstyleliteralemphasis{\sphinxupquote{List}}\sphinxstyleliteralemphasis{\sphinxupquote{{[}}}\sphinxstyleliteralemphasis{\sphinxupquote{float}}\sphinxstyleliteralemphasis{\sphinxupquote{{]}}}\sphinxstyleliteralemphasis{\sphinxupquote{{]} or }}\sphinxstyleliteralemphasis{\sphinxupquote{numpy.ndarray}}) \textendash{} The initial sugar (sucrose) concentration in the phloem
(\(\frac{mol}{m^3}\))

\item {} 
\sphinxstyleliteralstrong{\sphinxupquote{sugar\_loading\_profile}} (\sphinxstyleliteralemphasis{\sphinxupquote{List}}\sphinxstyleliteralemphasis{\sphinxupquote{{[}}}\sphinxstyleliteralemphasis{\sphinxupquote{List}}\sphinxstyleliteralemphasis{\sphinxupquote{{[}}}\sphinxstyleliteralemphasis{\sphinxupquote{float}}\sphinxstyleliteralemphasis{\sphinxupquote{{]}}}\sphinxstyleliteralemphasis{\sphinxupquote{{]} or }}\sphinxstyleliteralemphasis{\sphinxupquote{numpy.ndarray}}) \textendash{} the rate at which sugar concentration increases
in each phloem element (\(\frac{mol}{s}\))

\item {} 
\sphinxstyleliteralstrong{\sphinxupquote{sugar\_unloading\_profile}} (\sphinxstyleliteralemphasis{\sphinxupquote{List}}\sphinxstyleliteralemphasis{\sphinxupquote{{[}}}\sphinxstyleliteralemphasis{\sphinxupquote{float}}\sphinxstyleliteralemphasis{\sphinxupquote{{]} or }}\sphinxstyleliteralemphasis{\sphinxupquote{numpy.ndarray}}) \textendash{} The initial sugar unloading rate (the rate at which the
sugar concentration decreases in a given phloem element) (\(\frac{mol}{s}\)). The unloading rate is
updated in \sphinxhref{index.html\#src.odefun.odefun}{src.odefun.odefun}.

\item {} 
\sphinxstyleliteralstrong{\sphinxupquote{sugar\_target\_concentration}} (\sphinxstyleliteralemphasis{\sphinxupquote{float}}) \textendash{} the target concentration after which the sugar unloading
starts (\(\frac{mol}{m^3}\))

\item {} 
\sphinxstyleliteralstrong{\sphinxupquote{sugar\_unloading\_slope}} (\sphinxstyleliteralemphasis{\sphinxupquote{float}}) \textendash{} 
the slope parameter for unloading (see
\sphinxhref{https://academic.oup.com/aob/article/114/4/653/2769025}{Nikinmaa et. al., (2014)}).


\item {} 
\sphinxstyleliteralstrong{\sphinxupquote{axial\_permeability\_profile}} (\sphinxstyleliteralemphasis{\sphinxupquote{List}}\sphinxstyleliteralemphasis{\sphinxupquote{{[}}}\sphinxstyleliteralemphasis{\sphinxupquote{List}}\sphinxstyleliteralemphasis{\sphinxupquote{{[}}}\sphinxstyleliteralemphasis{\sphinxupquote{float}}\sphinxstyleliteralemphasis{\sphinxupquote{{]}}}\sphinxstyleliteralemphasis{\sphinxupquote{{]} or }}\sphinxstyleliteralemphasis{\sphinxupquote{numpy.ndarray}}) \textendash{} axial permeabilities of both xylem and phloem
(\(m^2\))

\item {} 
\sphinxstyleliteralstrong{\sphinxupquote{radial\_hydraulic\_conductivity\_profile}} (\sphinxstyleliteralemphasis{\sphinxupquote{List}}\sphinxstyleliteralemphasis{\sphinxupquote{{[}}}\sphinxstyleliteralemphasis{\sphinxupquote{float}}\sphinxstyleliteralemphasis{\sphinxupquote{{]}}}\sphinxstyleliteralemphasis{\sphinxupquote{{]} or }}\sphinxstyleliteralemphasis{\sphinxupquote{numpy.ndarray}}) \textendash{} radial hydraulic conductivity between the
xylem and the phloem (\(\frac{m}{Pa \: s}\))

\item {} 
\sphinxstyleliteralstrong{\sphinxupquote{elastic\_modulus\_profile}} (\sphinxstyleliteralemphasis{\sphinxupquote{List}}\sphinxstyleliteralemphasis{\sphinxupquote{{[}}}\sphinxstyleliteralemphasis{\sphinxupquote{List}}\sphinxstyleliteralemphasis{\sphinxupquote{{[}}}\sphinxstyleliteralemphasis{\sphinxupquote{float}}\sphinxstyleliteralemphasis{\sphinxupquote{{]}}}\sphinxstyleliteralemphasis{\sphinxupquote{{]} or }}\sphinxstyleliteralemphasis{\sphinxupquote{numpy.ndarray}}) \textendash{} Elastic modulus of every element (\(Pa\)).

\item {} 
\sphinxstyleliteralstrong{\sphinxupquote{ground\_water\_potential}} (\sphinxstyleliteralemphasis{\sphinxupquote{float}}) \textendash{} The water potential in the soil. This is used to calculate the sap flux between
soil and the bottom xylem element.

\end{itemize}

\end{description}\end{quote}
\index{height (src.tree.Tree attribute)@\spxentry{height}\spxextra{src.tree.Tree attribute}}

\begin{fulllineitems}
\phantomsection\label{\detokenize{index:src.tree.Tree.height}}\pysigline{\sphinxbfcode{\sphinxupquote{height}}}
total tree height (\(m\))
\begin{quote}\begin{description}
\item[{Type}] \leavevmode
float

\end{description}\end{quote}

\end{fulllineitems}

\index{num\_elements (src.tree.Tree attribute)@\spxentry{num\_elements}\spxextra{src.tree.Tree attribute}}

\begin{fulllineitems}
\phantomsection\label{\detokenize{index:src.tree.Tree.num_elements}}\pysigline{\sphinxbfcode{\sphinxupquote{num\_elements}}}
number of vertical elemenets in the tree.
\begin{quote}\begin{description}
\item[{Type}] \leavevmode
float

\end{description}\end{quote}

\end{fulllineitems}

\index{transpiration\_rate (src.tree.Tree attribute)@\spxentry{transpiration\_rate}\spxextra{src.tree.Tree attribute}}

\begin{fulllineitems}
\phantomsection\label{\detokenize{index:src.tree.Tree.transpiration_rate}}\pysigline{\sphinxbfcode{\sphinxupquote{transpiration\_rate}}}
The rate of transpiration
(\(\frac{kg}{s}\)) in the xylem.
\begin{quote}\begin{description}
\item[{Type}] \leavevmode
numpy.ndarray(dtype=float, ndim=2) {[}tree.num\_elements, 1{]}

\end{description}\end{quote}

\end{fulllineitems}

\index{photosynthesis\_rate (src.tree.Tree attribute)@\spxentry{photosynthesis\_rate}\spxextra{src.tree.Tree attribute}}

\begin{fulllineitems}
\phantomsection\label{\detokenize{index:src.tree.Tree.photosynthesis_rate}}\pysigline{\sphinxbfcode{\sphinxupquote{photosynthesis\_rate}}}
The rate of photosynthesis
(\(\frac{mol}{s}\)). Currently this variable is not used.
\begin{quote}\begin{description}
\item[{Type}] \leavevmode
numpy.ndarray(dtype=float, ndim=2) {[}tree.num\_elements, 1{]}

\end{description}\end{quote}

\end{fulllineitems}

\index{sugar\_loading\_rate (src.tree.Tree attribute)@\spxentry{sugar\_loading\_rate}\spxextra{src.tree.Tree attribute}}

\begin{fulllineitems}
\phantomsection\label{\detokenize{index:src.tree.Tree.sugar_loading_rate}}\pysigline{\sphinxbfcode{\sphinxupquote{sugar\_loading\_rate}}}
The rate at which sugar
concentration increases in each phloem element (\(\frac{mol}{s}\)).
\begin{quote}\begin{description}
\item[{Type}] \leavevmode
numpy.ndarray(dtype=float, ndim=2) {[}tree.num\_elements, 1{]}

\end{description}\end{quote}

\end{fulllineitems}

\index{sugar\_unloading\_rate (src.tree.Tree attribute)@\spxentry{sugar\_unloading\_rate}\spxextra{src.tree.Tree attribute}}

\begin{fulllineitems}
\phantomsection\label{\detokenize{index:src.tree.Tree.sugar_unloading_rate}}\pysigline{\sphinxbfcode{\sphinxupquote{sugar\_unloading\_rate}}}
The rate at which the
sugar concentration decreases in a given phloem element (\(\frac{mol}{s}\)).
\begin{quote}\begin{description}
\item[{Type}] \leavevmode
numpy.ndarray(dtype=float, ndim=2) {[}tree.num\_elements, 1{]}

\end{description}\end{quote}

\end{fulllineitems}

\index{sugar\_target\_concentration (src.tree.Tree attribute)@\spxentry{sugar\_target\_concentration}\spxextra{src.tree.Tree attribute}}

\begin{fulllineitems}
\phantomsection\label{\detokenize{index:src.tree.Tree.sugar_target_concentration}}\pysigline{\sphinxbfcode{\sphinxupquote{sugar\_target\_concentration}}}
The target concentration after which the sugar unloading
starts (\(\frac{mol}{m^3}\)).
\begin{quote}\begin{description}
\item[{Type}] \leavevmode
float

\end{description}\end{quote}

\end{fulllineitems}

\index{sugar\_unloading\_slope (src.tree.Tree attribute)@\spxentry{sugar\_unloading\_slope}\spxextra{src.tree.Tree attribute}}

\begin{fulllineitems}
\phantomsection\label{\detokenize{index:src.tree.Tree.sugar_unloading_slope}}\pysigline{\sphinxbfcode{\sphinxupquote{sugar\_unloading\_slope}}}
The slope parameter for unloading (see
{[}Nikinmaa et. al., (2014){]}(\sphinxurl{https://academic.oup.com/aob/article/114/4/653/2769025})).
\begin{quote}\begin{description}
\item[{Type}] \leavevmode
float

\end{description}\end{quote}

\end{fulllineitems}

\index{solutes (src.tree.Tree attribute)@\spxentry{solutes}\spxextra{src.tree.Tree attribute}}

\begin{fulllineitems}
\phantomsection\label{\detokenize{index:src.tree.Tree.solutes}}\pysigline{\sphinxbfcode{\sphinxupquote{solutes}}}
Array of
src.solute.Solute which contain the solutes in the sap of xylem and phloem.
\begin{quote}\begin{description}
\item[{Type}] \leavevmode
numpy.ndarray(dtype=src.solute.Solute, ndim=2) {[}tree.num\_elements, 2{]}

\end{description}\end{quote}

\end{fulllineitems}

\index{axial\_permeability (src.tree.Tree attribute)@\spxentry{axial\_permeability}\spxextra{src.tree.Tree attribute}}

\begin{fulllineitems}
\phantomsection\label{\detokenize{index:src.tree.Tree.axial_permeability}}\pysigline{\sphinxbfcode{\sphinxupquote{axial\_permeability}}}
Axial permeabilities of both
xylem and phloem (\(m^2\)).
\begin{quote}\begin{description}
\item[{Type}] \leavevmode
numpy.ndarray(dtype=float, ndim=2) {[}tree.num\_elements, 2{]}

\end{description}\end{quote}

\end{fulllineitems}

\index{radial\_hydraulic\_conductivity (src.tree.Tree attribute)@\spxentry{radial\_hydraulic\_conductivity}\spxextra{src.tree.Tree attribute}}

\begin{fulllineitems}
\phantomsection\label{\detokenize{index:src.tree.Tree.radial_hydraulic_conductivity}}\pysigline{\sphinxbfcode{\sphinxupquote{radial\_hydraulic\_conductivity}}}
Radial hydraulic
conductivity between the xylem and the phloem (\(\frac{m}{Pa \: s}\)).
\begin{quote}\begin{description}
\item[{Type}] \leavevmode
numpy.ndarray(dtype=float, ndim=2) {[}tree.num\_elements, 1{]}

\end{description}\end{quote}

\end{fulllineitems}

\index{elastic\_modulus (src.tree.Tree attribute)@\spxentry{elastic\_modulus}\spxextra{src.tree.Tree attribute}}

\begin{fulllineitems}
\phantomsection\label{\detokenize{index:src.tree.Tree.elastic_modulus}}\pysigline{\sphinxbfcode{\sphinxupquote{elastic\_modulus}}}
Elastic modulus of every element
(\(Pa\)).
\begin{quote}\begin{description}
\item[{Type}] \leavevmode
numpy.ndarray(dtype=float, ndim=2) {[}tree.num\_elements, 2{]}

\end{description}\end{quote}

\end{fulllineitems}

\index{ground\_water\_potential (src.tree.Tree attribute)@\spxentry{ground\_water\_potential}\spxextra{src.tree.Tree attribute}}

\begin{fulllineitems}
\phantomsection\label{\detokenize{index:src.tree.Tree.ground_water_potential}}\pysigline{\sphinxbfcode{\sphinxupquote{ground\_water\_potential}}}
The water potential in the soil.
\begin{quote}\begin{description}
\item[{Type}] \leavevmode
float

\end{description}\end{quote}

\end{fulllineitems}

\index{pressure (src.tree.Tree attribute)@\spxentry{pressure}\spxextra{src.tree.Tree attribute}}

\begin{fulllineitems}
\phantomsection\label{\detokenize{index:src.tree.Tree.pressure}}\pysigline{\sphinxbfcode{\sphinxupquote{pressure}}}
Pressure of each element (\(Pa\))
\begin{quote}\begin{description}
\item[{Type}] \leavevmode
numpy.ndarray(dtype=float, ndim=2) {[}tree.num\_elements, 2{]}

\end{description}\end{quote}

\end{fulllineitems}

\index{element\_radius (src.tree.Tree attribute)@\spxentry{element\_radius}\spxextra{src.tree.Tree attribute}}

\begin{fulllineitems}
\phantomsection\label{\detokenize{index:src.tree.Tree.element_radius}}\pysigline{\sphinxbfcode{\sphinxupquote{element\_radius}}}
Radius of each element (\(m\))
\begin{quote}\begin{description}
\item[{Type}] \leavevmode
numpy.ndarray(dtype=float, ndim=2) {[}tree.num\_elements, 2{]}

\end{description}\end{quote}

\end{fulllineitems}

\index{element\_height (src.tree.Tree attribute)@\spxentry{element\_height}\spxextra{src.tree.Tree attribute}}

\begin{fulllineitems}
\phantomsection\label{\detokenize{index:src.tree.Tree.element_height}}\pysigline{\sphinxbfcode{\sphinxupquote{element\_height}}}
Height of each element (\(m\))
\begin{quote}\begin{description}
\item[{Type}] \leavevmode
numpy.ndarray(dtype=float, ndim=2) {[}tree.num\_elements, 2{]}

\end{description}\end{quote}

\end{fulllineitems}

\index{viscosity (src.tree.Tree attribute)@\spxentry{viscosity}\spxextra{src.tree.Tree attribute}}

\begin{fulllineitems}
\phantomsection\label{\detokenize{index:src.tree.Tree.viscosity}}\pysigline{\sphinxbfcode{\sphinxupquote{viscosity}}}
The dynamic viscosity of each element
(\(Pa \: s\))
\begin{quote}\begin{description}
\item[{Type}] \leavevmode
numpy.ndarray(dtype=float, ndim=2) {[}tree.num\_elements, 2{]}

\end{description}\end{quote}

\end{fulllineitems}

\index{cross\_sectional\_area() (src.tree.Tree method)@\spxentry{cross\_sectional\_area()}\spxextra{src.tree.Tree method}}

\begin{fulllineitems}
\phantomsection\label{\detokenize{index:src.tree.Tree.cross_sectional_area}}\pysiglinewithargsret{\sphinxbfcode{\sphinxupquote{cross\_sectional\_area}}}{\emph{\DUrole{n}{ind}\DUrole{p}{:} \DUrole{n}{List\DUrole{p}{{[}}int\DUrole{p}{{]}}} \DUrole{o}{=} \DUrole{default_value}{None}}}{{ $\rightarrow$ numpy.ndarray}}
Calculates the cross\sphinxhyphen{}sectional area between the xylemn and the phloem.

The cross sectional area is equal to lateral surface area of the xylem.
\begin{quote}\begin{description}
\item[{Parameters}] \leavevmode
\sphinxstyleliteralstrong{\sphinxupquote{ind}} (\sphinxstyleliteralemphasis{\sphinxupquote{List}}\sphinxstyleliteralemphasis{\sphinxupquote{{[}}}\sphinxstyleliteralemphasis{\sphinxupquote{int}}\sphinxstyleliteralemphasis{\sphinxupquote{{]} or }}\sphinxstyleliteralemphasis{\sphinxupquote{numpy.ndarray}}\sphinxstyleliteralemphasis{\sphinxupquote{(}}\sphinxstyleliteralemphasis{\sphinxupquote{dtype=int}}\sphinxstyleliteralemphasis{\sphinxupquote{, }}\sphinxstyleliteralemphasis{\sphinxupquote{ndim=1}}\sphinxstyleliteralemphasis{\sphinxupquote{)}}\sphinxstyleliteralemphasis{\sphinxupquote{, }}\sphinxstyleliteralemphasis{\sphinxupquote{optional}}) \textendash{} the indices of the elements
for which the cross\sphinxhyphen{}sectinoal area is calculated. If no ind is given, the
cross\sphinxhyphen{}sectional area is calculated for every element.

\item[{Returns}] \leavevmode
Cross\sphinxhyphen{}sectional area
between the xylem and phloem elements (\(m^2\))

\item[{Return type}] \leavevmode
numpy.ndarray(dtype=float, ndim=2) {[}len(ind) or self.num\_elements, 1{]}

\end{description}\end{quote}

\end{fulllineitems}

\index{element\_area() (src.tree.Tree method)@\spxentry{element\_area()}\spxextra{src.tree.Tree method}}

\begin{fulllineitems}
\phantomsection\label{\detokenize{index:src.tree.Tree.element_area}}\pysiglinewithargsret{\sphinxbfcode{\sphinxupquote{element\_area}}}{\emph{\DUrole{n}{ind}\DUrole{p}{:} \DUrole{n}{List\DUrole{p}{{[}}int\DUrole{p}{{]}}} \DUrole{o}{=} \DUrole{default_value}{None}}, \emph{\DUrole{n}{column}\DUrole{p}{:} \DUrole{n}{int} \DUrole{o}{=} \DUrole{default_value}{0}}}{{ $\rightarrow$ numpy.ndarray}}
Calculates the base area of the xylem or the phloem.
\begin{quote}\begin{description}
\item[{Parameters}] \leavevmode\begin{itemize}
\item {} 
\sphinxstyleliteralstrong{\sphinxupquote{ind}} (\sphinxstyleliteralemphasis{\sphinxupquote{List}}\sphinxstyleliteralemphasis{\sphinxupquote{{[}}}\sphinxstyleliteralemphasis{\sphinxupquote{int}}\sphinxstyleliteralemphasis{\sphinxupquote{{]} or }}\sphinxstyleliteralemphasis{\sphinxupquote{numpy.ndarray}}\sphinxstyleliteralemphasis{\sphinxupquote{(}}\sphinxstyleliteralemphasis{\sphinxupquote{dtype=int}}\sphinxstyleliteralemphasis{\sphinxupquote{, }}\sphinxstyleliteralemphasis{\sphinxupquote{ndim=1}}\sphinxstyleliteralemphasis{\sphinxupquote{)}}\sphinxstyleliteralemphasis{\sphinxupquote{, }}\sphinxstyleliteralemphasis{\sphinxupquote{optional}}) \textendash{} the indices of the elements
for which the base area is calculated. If no ind is given, the base area is calculated
for every element.

\item {} 
\sphinxstyleliteralstrong{\sphinxupquote{column}} (\sphinxstyleliteralemphasis{\sphinxupquote{int}}\sphinxstyleliteralemphasis{\sphinxupquote{, }}\sphinxstyleliteralemphasis{\sphinxupquote{optional}}) \textendash{} The column in the tree grid for which the base area is calculated.
use column=0 for the xylem and column=1 for the phloem. If not column is given returns
the base area for the xylem.

\end{itemize}

\item[{Returns}] \leavevmode
Base area of either
the xylem or the phloem (\(m^2\))

\item[{Return type}] \leavevmode
numpy.ndarray(dtype=float, ndim=2) {[}len(ind) or self.num\_elements, 1{]}

\end{description}\end{quote}

\end{fulllineitems}

\index{element\_volume() (src.tree.Tree method)@\spxentry{element\_volume()}\spxextra{src.tree.Tree method}}

\begin{fulllineitems}
\phantomsection\label{\detokenize{index:src.tree.Tree.element_volume}}\pysiglinewithargsret{\sphinxbfcode{\sphinxupquote{element\_volume}}}{\emph{\DUrole{n}{ind}\DUrole{p}{:} \DUrole{n}{List\DUrole{p}{{[}}int\DUrole{p}{{]}}} \DUrole{o}{=} \DUrole{default_value}{None}}, \emph{\DUrole{n}{column}\DUrole{p}{:} \DUrole{n}{int} \DUrole{o}{=} \DUrole{default_value}{0}}}{{ $\rightarrow$ numpy.ndarray}}
Calculates the volume of the xylem or the phloem.
\begin{quote}\begin{description}
\item[{Parameters}] \leavevmode\begin{itemize}
\item {} 
\sphinxstyleliteralstrong{\sphinxupquote{ind}} (\sphinxstyleliteralemphasis{\sphinxupquote{List}}\sphinxstyleliteralemphasis{\sphinxupquote{{[}}}\sphinxstyleliteralemphasis{\sphinxupquote{int}}\sphinxstyleliteralemphasis{\sphinxupquote{{]} or }}\sphinxstyleliteralemphasis{\sphinxupquote{numpy.ndarray}}\sphinxstyleliteralemphasis{\sphinxupquote{(}}\sphinxstyleliteralemphasis{\sphinxupquote{dtype=int}}\sphinxstyleliteralemphasis{\sphinxupquote{, }}\sphinxstyleliteralemphasis{\sphinxupquote{ndim=1}}\sphinxstyleliteralemphasis{\sphinxupquote{)}}\sphinxstyleliteralemphasis{\sphinxupquote{, }}\sphinxstyleliteralemphasis{\sphinxupquote{optional}}) \textendash{} the indices of the elements
for which the volume is calculated. If no ind is given, the volume is calculated
for every element.

\item {} 
\sphinxstyleliteralstrong{\sphinxupquote{column}} (\sphinxstyleliteralemphasis{\sphinxupquote{int}}\sphinxstyleliteralemphasis{\sphinxupquote{, }}\sphinxstyleliteralemphasis{\sphinxupquote{optional}}) \textendash{} The column in the tree grid for which the volume is calculated.
use column=0 for the xylem and column=1 for the phloem. If not column is given returns
the volume for the xylem.

\end{itemize}

\item[{Returns}] \leavevmode
Volume of either
the xylem or the phloem (\(m^3\))

\item[{Return type}] \leavevmode
numpy.ndarray(dtype=float, ndim=2) {[}len(ind) or self.num\_elements, 1{]}

\end{description}\end{quote}

\end{fulllineitems}

\index{sugar\_concentration\_as\_numpy\_array() (src.tree.Tree method)@\spxentry{sugar\_concentration\_as\_numpy\_array()}\spxextra{src.tree.Tree method}}

\begin{fulllineitems}
\phantomsection\label{\detokenize{index:src.tree.Tree.sugar_concentration_as_numpy_array}}\pysiglinewithargsret{\sphinxbfcode{\sphinxupquote{sugar\_concentration\_as\_numpy\_array}}}{}{{ $\rightarrow$ numpy.ndarray}}
Transforms the phloem sugar concentration in \sphinxhref{index.html\#src.tree.Tree.solutes}{self.solutes}
into numpy.ndarray.
\begin{quote}\begin{description}
\item[{Returns}] \leavevmode
The sugar concentration in the phloem.
(\(\frac{mol}{m^3}\))

\item[{Return type}] \leavevmode
numpy.ndarray(dtype=float, ndim=2) {[}self.num\_elements, 1{]}

\end{description}\end{quote}

\end{fulllineitems}

\index{update\_sap\_viscosity() (src.tree.Tree method)@\spxentry{update\_sap\_viscosity()}\spxextra{src.tree.Tree method}}

\begin{fulllineitems}
\phantomsection\label{\detokenize{index:src.tree.Tree.update_sap_viscosity}}\pysiglinewithargsret{\sphinxbfcode{\sphinxupquote{update\_sap\_viscosity}}}{}{{ $\rightarrow$ None}}
Calculates and sets the viscosity in the phloem according to the sugar concenration.

The sap viscosity is calculated according to Morrison (2002)
\begin{equation*}
\begin{split}\eta = \eta_w \exp{\frac{4.68 \cdot 0.956 \Phi_s}{1-0.956 \Phi_s}}\end{split}
\end{equation*}\begin{description}
\item[{where}] \leavevmode
\(\eta_w\): Dynamic viscosity of water (\(\eta_w \approx 0.001\))
\(\Phi_s\): Volume fraction of sugar (sucrose) in the phloem sap.

\end{description}
\subsubsection*{References}

Morison, Ken R. “Viscosity equations for sucrose solutions: old and new 2002.”
Proceedings of the 9th APCChE Congress and CHEMECA. 2002.

\end{fulllineitems}

\index{update\_sugar\_concentration() (src.tree.Tree method)@\spxentry{update\_sugar\_concentration()}\spxextra{src.tree.Tree method}}

\begin{fulllineitems}
\phantomsection\label{\detokenize{index:src.tree.Tree.update_sugar_concentration}}\pysiglinewithargsret{\sphinxbfcode{\sphinxupquote{update\_sugar\_concentration}}}{\emph{\DUrole{n}{new\_concentration}\DUrole{p}{:} \DUrole{n}{numpy.ndarray}}}{{ $\rightarrow$ None}}
Sets the sugar concentration in \sphinxhref{index.html\#src.tree.Tree.solutes}{self.solutes} to new\_concentration.
\begin{quote}\begin{description}
\item[{Parameters}] \leavevmode
\sphinxstyleliteralstrong{\sphinxupquote{new\_concentration}} (\sphinxstyleliteralemphasis{\sphinxupquote{numpy.ndarray}}\sphinxstyleliteralemphasis{\sphinxupquote{(}}\sphinxstyleliteralemphasis{\sphinxupquote{dtype=float}}\sphinxstyleliteralemphasis{\sphinxupquote{, }}\sphinxstyleliteralemphasis{\sphinxupquote{ndim=2}}\sphinxstyleliteralemphasis{\sphinxupquote{)}}\sphinxstyleliteralemphasis{\sphinxupquote{{[}}}\sphinxstyleliteralemphasis{\sphinxupquote{self.num\_elements}}\sphinxstyleliteralemphasis{\sphinxupquote{,}}\sphinxstyleliteralemphasis{\sphinxupquote{1}}\sphinxstyleliteralemphasis{\sphinxupquote{{]}}}) \textendash{} new concentration values.
the order is from top of the tree (first element, new\_concentration{[}0{]}) to bottom of the tree
(last element, new\_concentration{[}self.num\_elements\sphinxhyphen{}1{]}) (\(\frac{mol}{m^3}\))

\end{description}\end{quote}

\end{fulllineitems}


\end{fulllineitems}

\index{Solute (class in src.solute)@\spxentry{Solute}\spxextra{class in src.solute}}

\begin{fulllineitems}
\phantomsection\label{\detokenize{index:src.solute.Solute}}\pysiglinewithargsret{\sphinxbfcode{\sphinxupquote{class }}\sphinxcode{\sphinxupquote{src.solute.}}\sphinxbfcode{\sphinxupquote{Solute}}}{\emph{\DUrole{n}{name}\DUrole{p}{:} \DUrole{n}{str}}, \emph{\DUrole{n}{molar\_mass}\DUrole{p}{:} \DUrole{n}{float}}, \emph{\DUrole{n}{density}\DUrole{p}{:} \DUrole{n}{float}}, \emph{\DUrole{n}{concentration}\DUrole{p}{:} \DUrole{n}{float}}}{}
Contains the variables to model a solute compound
\begin{quote}\begin{description}
\item[{Parameters}] \leavevmode\begin{itemize}
\item {} 
\sphinxstyleliteralstrong{\sphinxupquote{name}} (\sphinxstyleliteralemphasis{\sphinxupquote{str}}) \textendash{} Name of the compound

\item {} 
\sphinxstyleliteralstrong{\sphinxupquote{molar\_mass}} (\sphinxstyleliteralemphasis{\sphinxupquote{float}}) \textendash{} Molar mass of the compound (\(\frac{kg}{mol}\))

\item {} 
\sphinxstyleliteralstrong{\sphinxupquote{density}} \textendash{} Liquid phase density of the compound (\(\frac{kg}{m^3}\))

\item {} 
\sphinxstyleliteralstrong{\sphinxupquote{concentration}} \textendash{} Concentration of the compound in the sap solution (\(\frac{mol}{m^3}\))

\end{itemize}

\end{description}\end{quote}
\index{name (src.solute.Solute attribute)@\spxentry{name}\spxextra{src.solute.Solute attribute}}

\begin{fulllineitems}
\phantomsection\label{\detokenize{index:src.solute.Solute.name}}\pysigline{\sphinxbfcode{\sphinxupquote{name}}}
Name of the compound
\begin{quote}\begin{description}
\item[{Type}] \leavevmode
str

\end{description}\end{quote}

\end{fulllineitems}

\index{molar\_mass (src.solute.Solute attribute)@\spxentry{molar\_mass}\spxextra{src.solute.Solute attribute}}

\begin{fulllineitems}
\phantomsection\label{\detokenize{index:src.solute.Solute.molar_mass}}\pysigline{\sphinxbfcode{\sphinxupquote{molar\_mass}}}
Molar mass of the compound (\(\frac{kg}{mol}\))
\begin{quote}\begin{description}
\item[{Type}] \leavevmode
float

\end{description}\end{quote}

\end{fulllineitems}

\index{density (src.solute.Solute attribute)@\spxentry{density}\spxextra{src.solute.Solute attribute}}

\begin{fulllineitems}
\phantomsection\label{\detokenize{index:src.solute.Solute.density}}\pysigline{\sphinxbfcode{\sphinxupquote{density}}}
Liquid phase density of the compound (\(\frac{kg}{m^3}\))

\end{fulllineitems}

\index{concentration (src.solute.Solute attribute)@\spxentry{concentration}\spxextra{src.solute.Solute attribute}}

\begin{fulllineitems}
\phantomsection\label{\detokenize{index:src.solute.Solute.concentration}}\pysigline{\sphinxbfcode{\sphinxupquote{concentration}}}
Concentration of the compound in the sap solution (\(\frac{mol}{m^3}\))

\end{fulllineitems}


\end{fulllineitems}

\index{odefun() (in module src.odefun)@\spxentry{odefun()}\spxextra{in module src.odefun}}

\begin{fulllineitems}
\phantomsection\label{\detokenize{index:src.odefun.odefun}}\pysiglinewithargsret{\sphinxcode{\sphinxupquote{src.odefun.}}\sphinxbfcode{\sphinxupquote{odefun}}}{\emph{\DUrole{n}{t}}, \emph{\DUrole{n}{y}}, \emph{\DUrole{n}{model}}}{}
Calculates the right hand side of the model ODEs.

\end{fulllineitems}

\phantomsection\label{\detokenize{index:module-src.constants}}\index{module@\spxentry{module}!src.constants@\spxentry{src.constants}}\index{src.constants@\spxentry{src.constants}!module@\spxentry{module}}\phantomsection\label{\detokenize{index:module-src.model_variables}}\index{module@\spxentry{module}!src.model\_variables@\spxentry{src.model\_variables}}\index{src.model\_variables@\spxentry{src.model\_variables}!module@\spxentry{module}}\phantomsection\label{\detokenize{index:module-src.tools.iotools}}\index{module@\spxentry{module}!src.tools.iotools@\spxentry{src.tools.iotools}}\index{src.tools.iotools@\spxentry{src.tools.iotools}!module@\spxentry{module}}\index{initialize\_netcdf() (in module src.tools.iotools)@\spxentry{initialize\_netcdf()}\spxextra{in module src.tools.iotools}}

\begin{fulllineitems}
\phantomsection\label{\detokenize{index:src.tools.iotools.initialize_netcdf}}\pysiglinewithargsret{\sphinxcode{\sphinxupquote{src.tools.iotools.}}\sphinxbfcode{\sphinxupquote{initialize\_netcdf}}}{\emph{\DUrole{n}{model}}, \emph{\DUrole{n}{variables}\DUrole{p}{:} \DUrole{n}{Dict}}}{{ $\rightarrow$ netCDF4.\_netCDF4.Dataset}}
\end{fulllineitems}

\index{tree\_properties\_to\_dict() (in module src.tools.iotools)@\spxentry{tree\_properties\_to\_dict()}\spxextra{in module src.tools.iotools}}

\begin{fulllineitems}
\phantomsection\label{\detokenize{index:src.tools.iotools.tree_properties_to_dict}}\pysiglinewithargsret{\sphinxcode{\sphinxupquote{src.tools.iotools.}}\sphinxbfcode{\sphinxupquote{tree\_properties\_to\_dict}}}{\emph{\DUrole{n}{tree}\DUrole{p}{:} \DUrole{n}{{\hyperref[\detokenize{index:src.tree.Tree}]{\sphinxcrossref{src.tree.Tree}}}}}}{{ $\rightarrow$ Dict}}
cast tree properties to dictionary for saving

\end{fulllineitems}

\index{write\_netcdf() (in module src.tools.iotools)@\spxentry{write\_netcdf()}\spxextra{in module src.tools.iotools}}

\begin{fulllineitems}
\phantomsection\label{\detokenize{index:src.tools.iotools.write_netcdf}}\pysiglinewithargsret{\sphinxcode{\sphinxupquote{src.tools.iotools.}}\sphinxbfcode{\sphinxupquote{write\_netcdf}}}{\emph{\DUrole{n}{ncf}\DUrole{p}{:} \DUrole{n}{netCDF4.\_netCDF4.Dataset}}, \emph{\DUrole{n}{results}\DUrole{p}{:} \DUrole{n}{Dict}}}{{ $\rightarrow$ None}}
\end{fulllineitems}

\phantomsection\label{\detokenize{index:module-src.tools.plotting}}\index{module@\spxentry{module}!src.tools.plotting@\spxentry{src.tools.plotting}}\index{src.tools.plotting@\spxentry{src.tools.plotting}!module@\spxentry{module}}\index{plot\_phloem\_pressure\_top\_bottom() (in module src.tools.plotting)@\spxentry{plot\_phloem\_pressure\_top\_bottom()}\spxextra{in module src.tools.plotting}}

\begin{fulllineitems}
\phantomsection\label{\detokenize{index:src.tools.plotting.plot_phloem_pressure_top_bottom}}\pysiglinewithargsret{\sphinxcode{\sphinxupquote{src.tools.plotting.}}\sphinxbfcode{\sphinxupquote{plot\_phloem\_pressure\_top\_bottom}}}{\emph{\DUrole{n}{filename}\DUrole{p}{:} \DUrole{n}{str}}}{}
\end{fulllineitems}

\index{plot\_simulation\_results() (in module src.tools.plotting)@\spxentry{plot\_simulation\_results()}\spxextra{in module src.tools.plotting}}

\begin{fulllineitems}
\phantomsection\label{\detokenize{index:src.tools.plotting.plot_simulation_results}}\pysiglinewithargsret{\sphinxcode{\sphinxupquote{src.tools.plotting.}}\sphinxbfcode{\sphinxupquote{plot\_simulation\_results}}}{\emph{\DUrole{n}{filename}\DUrole{p}{:} \DUrole{n}{str}}, \emph{\DUrole{n}{foldername}\DUrole{p}{:} \DUrole{n}{str}}}{}
\end{fulllineitems}

\index{plot\_variable\_vs\_time() (in module src.tools.plotting)@\spxentry{plot\_variable\_vs\_time()}\spxextra{in module src.tools.plotting}}

\begin{fulllineitems}
\phantomsection\label{\detokenize{index:src.tools.plotting.plot_variable_vs_time}}\pysiglinewithargsret{\sphinxcode{\sphinxupquote{src.tools.plotting.}}\sphinxbfcode{\sphinxupquote{plot\_variable\_vs\_time}}}{\emph{\DUrole{n}{filename}\DUrole{p}{:} \DUrole{n}{str}}, \emph{\DUrole{n}{params}\DUrole{p}{:} \DUrole{n}{Dict} \DUrole{o}{=} \DUrole{default_value}{None}}}{}
\end{fulllineitems}

\index{plot\_xylem\_pressure\_top\_bottom() (in module src.tools.plotting)@\spxentry{plot\_xylem\_pressure\_top\_bottom()}\spxextra{in module src.tools.plotting}}

\begin{fulllineitems}
\phantomsection\label{\detokenize{index:src.tools.plotting.plot_xylem_pressure_top_bottom}}\pysiglinewithargsret{\sphinxcode{\sphinxupquote{src.tools.plotting.}}\sphinxbfcode{\sphinxupquote{plot\_xylem\_pressure\_top\_bottom}}}{\emph{\DUrole{n}{filename}\DUrole{p}{:} \DUrole{n}{str}}}{}
\end{fulllineitems}



\renewcommand{\indexname}{Python Module Index}
\begin{sphinxtheindex}
\let\bigletter\sphinxstyleindexlettergroup
\bigletter{m}
\item\relax\sphinxstyleindexentry{main}\sphinxstyleindexpageref{index:\detokenize{module-main}}
\indexspace
\bigletter{s}
\item\relax\sphinxstyleindexentry{src.constants}\sphinxstyleindexpageref{index:\detokenize{module-src.constants}}
\item\relax\sphinxstyleindexentry{src.model}\sphinxstyleindexpageref{index:\detokenize{module-src.model}}
\item\relax\sphinxstyleindexentry{src.model\_variables}\sphinxstyleindexpageref{index:\detokenize{module-src.model_variables}}
\item\relax\sphinxstyleindexentry{src.tools.iotools}\sphinxstyleindexpageref{index:\detokenize{module-src.tools.iotools}}
\item\relax\sphinxstyleindexentry{src.tools.plotting}\sphinxstyleindexpageref{index:\detokenize{module-src.tools.plotting}}
\item\relax\sphinxstyleindexentry{src.tree}\sphinxstyleindexpageref{index:\detokenize{module-src.tree}}
\end{sphinxtheindex}

\renewcommand{\indexname}{Index}
\printindex
\end{document}